%---------------------------------------------------------------------------------
%                西南交通大学研究生学位论文:结论
%---------------------------------------------------------------------------------
\chapter*{结\qquad{}论}
\addcontentsline{toc}{chapter}{结论}
本章将对本论文的工作进行总结,
并对后续进一步的研究问题进行讨论。
\section*{本文工作总结}
随着集成电路技术、通讯技术的飞速发展,
无人机航拍逐渐成了普通人都可以应用的一项技术。
由于设备价格低廉、未来5G时代万物互联的前景,
基于现有TCP/IP协议栈体系的无人机视频传输技术仍然具有很大前景。
由于无人机工作于室外空域,
面临的是典型的大尺度衰落信道环境,
丢包、错包频繁,
面向有线网络设计的标准TCP无法很好应对这种场景,
导致无人机视频传输吞吐率上不去、时延很大。
\par
本文首先分析了基于TCP协议的无人机视频传输所面临的问题,
查阅了大量文献,总结了前人做出的关于提高TCP在有损信道环境下的表现的工作。
重点分析论述了Sundararajan等人的工作,即TCP/NC。
在Raspberry Pi上实现了TCP/NC协议,
并对TCP/NC在有损信道环境下的表现做了测试、分析。
与前人工作不同的是,
在详细比较了Batch Coding和Pipeline Coding的优缺点后,
根据TCP协议的滑动窗口概念,
采用了Pipeline Coding编码方式实现网络编码。
\par
针对TCP/NC在无人机视频传输应用场景中所存在的问题,
本文对其做了进一步改进,实现增强TCP/NC协议,即E-TCP/NC。
提出了前向重传机制以应对一个RTO内的突发丢包;
提出了一种新的自适应冗余度算法以应对网络的变化;
对冗余包的发送时机给出改进以降低平均解码时延;
设计补偿重传机制以降低解码端解码矩阵维度和减小解码时延;
改进编码系数的设定以降低首部开销。
最后本文搭建了无人机视频传输实验平台,
测试E-TCP/NC对无人机视频传输性能的提升。

\section*{未来工作展望}
在下一步的研究和学习过程中,我们将在以下方面做出改进:
\begin{enumerate}[fullwidth,itemindent=2em,label=(\arabic*)]
	\item NC层的编解码过程耗费资源太大。如第\ref{sec:monidiubaoceshijieguo}小节所述,100M网卡在添加了编解码模块后,峰值速度直接降到了12Mbps。这其中有代码优化原因,也有Raspberry Pi性能有限的原因,但本质上是由于编解码本来就耗费资源,而本文又以模块的形式实现E-TCP/NC,导致软中断时间过长,影响最终吞吐率。在今后的工作中,将优化目前代码,寻求新的编解码方式,以便可以同时处理多个字节,降低解码端的开销。
	\item 目前对TCP报文的处理仅仅覆盖了一部分情况,对于TCP协议的选项字段、URG字段、PSH字段等都未做处理。后期需要改进目前的实现,使E-TCP/NC完全兼容目前的TCP协议。
\end{enumerate}