%---------------------------------------------------------------------------------
%                西南交通大学研究生学位论文基本信息
%---------------------------------------------------------------------------------

% 论文信息
% 定义申请学位
% 请注意:博士学位候选人输入 Doctor,硕士学位候选人输入 Master
\def\degree{Master} 

% 论文中文标题
% 请注意:标题控制在36个汉字以内),其中用 \\ 实现标题换行,单行不得超过18个汉字。
% 		  \underline命令用于实现标题的下划线,请把标题输入在\underline命令的的{}中
\cTitle{\underline{面向无线网络的新型传输层} \\ \underline{协议设计与实现}}

% 论文英文标题(72个字符以内,包含空格)
\eTitle{Design and Implementation of New Transport Layer Protocol for Wireless Networks}


% 国内图书分类号
\CI{TM30}
% 国际图书分类号
\UDC{621.3}
% 保密等级
\secLevel{公开}

%----------------------------
% 学生信息
% 中英文姓名
\author{董泽锋}
\eAuthor{Dong Zefeng}

% 年级
\grade{2014}

% 学科(工学、理学、社会学,etc)
\cDiscipline{工学}
\eDiscipline{Engineering}

% 专业(10个汉字以内)
% 请注意:对于超过10个汉字的专业,比如“防灾减灾工程及防护工程”,目前排版仍然不够美观,正在修正。
%        也请相应专业的同学提供过往师兄的硕士论文以供参考。
\cMajor{通信与信息系统}
\eMajor{Communication and Information System}

% 导师(10个汉字以内)
% 请注意:导师姓名和导师的职称之间加上\hspace{0.1em}以增大间距,整体显得更为美观。 
\cTutor{陈庆春 \hspace{0.1em} 教授/博导}
\eTutor{Prof. Chen Qingchun}

% 封面日期,依次为年,月,日
\cDate{二〇一七}{四}{十三}
\eDate{2017}{April}{13}