%---------------------------------------------------------------------------------
%                学位论文LaTeX模板使用宏包文件
%---------------------------------------------------------------------------------

% 数学公式常用宏包(实现更多样的数学公式输入)
\usepackage{amsmath}
\newtheorem{myDef}{\hspace{2em}定义}[chapter]
\usepackage{mathtools}
\usepackage{wasysym}

\makeatletter  
\newif\if@restonecol  
\makeatother  
\let\algorithm\relax  
\let\endalgorithm\relax  
\usepackage[linesnumbered,ruled,vlined]{algorithm2e}%[ruled,vlined]{  
\usepackage{algpseudocode}  
\usepackage{amsmath}  

%\renewcommand{\algorithmicrequire}{\textbf{In:}}  % Use Input in the format of Algorithm  
%\renewcommand{\algorithmicensure}{\textbf{Out:}} % Use Output in the format of Algorithm   
\newcommand{\Break}{\textbf{break}}
\renewcommand{\Return}{\textbf{return}}
%\usepackage{algorithmicx}
%\renewcommand{\algorithmcfname}{算法}
\usepackage{colortbl}

\usepackage{listings}
\usepackage{float}
\renewcommand{\lstlistingname}{代码}
\usepackage{xcolor}
\lstset{
	basicstyle=\ttfamily,
	numbers=left, 
	%numberstyle=\tiny, 
	keywordstyle=\color{blue!70},  
	commentstyle=\color{red!50!green!50!blue!50}, 
	frame=shadowbox, 
	rulesepcolor=\color{red!20!green!20!blue!20},
	tabsize=4,breaklines 
}


\usepackage{booktabs}
\usepackage{threeparttable}
\usepackage{amssymb}

\usepackage{amsfonts}
% 表格制作常用宏包(实现更多表格功能,比如不等线粗的三线表)
\usepackage{multirow}
\usepackage{booktabs}

% 插入图片常用宏包(实现多种格式的调用)
\usepackage{graphicx}

% 图表题注格式宏包(实现《规范》要求的题注格式)
\usepackage{caption}
\captionsetup{labelformat=simple, labelsep=space, font=bf}

% TeX系列标识的正确输入宏包(实现正确插入TeX相关的字符串)
\usepackage{dtklogos}

% 列表制作常用宏包(用于调用小间距列表)
\usepackage{paralist}
\usepackage{tabularx}
\usepackage{array}
% 外框宏包
\usepackage{framed}
\usepackage{rotating}
\usepackage{url}

% 引用文献实[Ni-Ni+j]的宏包
\usepackage{cite}

% 使用列表时要用到的包
\usepackage{enumitem}
% 英文采用Times New Rome字体(建议把这个宏包放在最后避免发生宏包冲突)
\usepackage{fontspec}
\setmainfont{Times New Roman}