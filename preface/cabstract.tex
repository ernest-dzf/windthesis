%---------------------------------------------------------------------------------
%                西南交通大学研究生学位论文:中文摘要
%---------------------------------------------------------------------------------
% 中文摘要输入区
\cabstract
{
	%%写完对照摘要要求,看是否合适
	%%可以写本文的主要工作包括。。。。
	\par
	传输层协议(TCP)将丢包看做是链路出现拥塞的标志,然后启动相应的拥塞控制算法,降低传输速率。
	考虑到有线网络中极低的误码率,这一机制是适用的。
	\par
	然而在无线网络中,由于天气、障碍物、多径干扰等各种因素,其丢包率比在有线网络中大得多,且丢包多由传输错误导致。传统TCP无法鉴别拥塞丢包和非拥塞丢包,不分区别地频繁启动拥塞控制机制,降低数据发送速率,将导致TCP无法充分利用可用带宽。网络编码的出现对于解决这一问题给出了新的思路。通过在发送端冗余编码,可以掩盖链路中出现的随机丢包,让TCP只会检测到拥塞丢包,从而提高网络的吞吐率。
	\par
	本文分析了传统TCP协议在无线网络中所面临的问题,对近些年来改善无线网络中TCP性能的相关工作进行了梳理。
	介绍了网络编码原理,尤其是随机线性网络编码。对于batch-coding和pipeline-coding进行了比较,分析了其应用场景和优缺点。
	在以上基础上,结合TCP协议特点,如滑动窗口,设计实现了编码TCP协议,并将其移植入嵌入式设备。在真实丢包网络中,对编码TCP进行性能测试,与TCP-vegas协议进行性能对比,分析影响其性能的关键因素。在了解现有编码TCP协议不足的情形下,对其做了进一步改进。提出了一种新的自适应冗余算法,用于更好地应对网络状况的波动,适应不同丢包率的网络;借鉴MAC层的ARQ协议,设计了编码层的反馈重传机制,可以有效减少解码时延,减小编码端缓存队列长度,抵抗突发丢包。
	\par
	
}
{网络编码;传输层协议;重传;自适应冗余;丢包} 	% 中文关键词输入区,关键词之间用“;”分开,不超过五个