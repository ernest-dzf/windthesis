%---------------------------------------------------------------------------------
%                西南交通大学研究生学位论文:中文摘要
%---------------------------------------------------------------------------------
% 中文摘要输入区
\cabstract
{
	%%写完对照摘要要求,看是否合适
	%%可以写本文的主要工作包括。。。。
	无人机一般工作在室外空域,具有移动性高的特点,这决定了其经常面临大尺度衰落信道环境,数据链路错包、丢包频繁。因此,基于传输层协议( TCP ) 的无人机视频传输面临着吞吐率受限、视频不流畅的问题。这一问题很大程度上又是因为TCP协议本身的设计导致的。
	\par
	TCP将丢包看做是链路出现拥塞的标志,然后启动相应的拥塞控制算法,降低传输速率。
	考虑到有线网络中极低的误码率,这一机制是适用的。
	\par
	然而对于无人机来说,由于天气、障碍物、多径干扰等各种因素,TCP丢包率比在有线网络中大得多,且丢包多由传输错误导致。传统TCP无法鉴别拥塞丢包和非拥塞丢包,不分区别地频繁启动拥塞控制算法,降低数据发送速率,将导致TCP无法充分利用可用带宽,进而影响无人机视频传输的速度。网络编码的出现对于解决这一问题给出了一个思路。通过在发送端冗余编码,可以掩盖链路中出现的随机丢包,让TCP只会检测到拥塞丢包,从而提高数据传输的速度。
	\par
	本文分析了基于TCP协议的无人机视频传输所面临的问题,并指出问题核心是TCP协议在面对无线网络环境时的低效。对近些年来改善无线网络中TCP性能的相关工作进行了梳理。
	介绍了网络编码原理,尤其是随机线性网络编码。对于batch-coding和pipeline-coding进行了比较,分析了其应用场景和优缺点。
	在以上基础上,结合TCP协议特点,如滑动窗口,设计实现了流水线编码TCP协议,并将其移植入嵌入式设备。在模拟丢包网络中,对流水线编码TCP进行性能测试,与TCP-Reno协议进行性能对比,分析影响其性能的关键因素。在了解流水线编码TCP协议不足的情形下,对其做了进一步改进。提出了前向重传机制以应对一个RTO内的突发丢包;提出了一种新的自适应冗余度算法以应对网络的变化;对冗余包的发送时机给出改进以降低平均解码时延;设计补偿重传机制以降低解码端解码矩阵维度和减小解码时延;改进编码系数的设定以降低首部开销。最后,本文搭建了无人机视频传输实验平台,验证改进后的编码TCP协议对无人机视频传输数据吞吐率的提高。
	\par
	
}
{无人机视频传输;传输层协议;编码;重传;自适应冗余;丢包} 	% 中文关键词输入区,关键词之间用“;”分开,不超过五个