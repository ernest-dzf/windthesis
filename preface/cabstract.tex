%---------------------------------------------------------------------------------
%                西南交通大学研究生学位论文:中文摘要
%---------------------------------------------------------------------------------
% 中文摘要输入区
\cabstract
{
	%%写完对照摘要要求,看是否合适
	%%可以写本文的主要工作包括。。。。
无人机一般工作在室外空域,具有移动性高的特点,这决定了其经常面临大尺度衰落信道环境,数据链路错包、丢包频繁。因此,基于传输层协议(TCP)的无人机视频传输面临着吞吐率受限、时延较大的问题。这一问题很大程度上又是因为TCP协议本身的设计导致的。
	\par
TCP协议中,接收端通过反馈ACK来告诉发送端某个报文的丢失,然后发送端据此重传相应的报文。对于无人机视频传输等实时性要求很高的应用来说,丢包反馈重传耗时太长。不仅如此,TCP将丢包看做是链路出现拥塞的标志,然后启动相应的拥塞控制算法,降低传输速率。考虑到有线网络中极低的误码率、拥塞是丢包的主要原因,这一机制是适用的。然而对于无人机来说,由于天气、障碍物、多径干扰等各种因素,TCP丢包率比在有线网络中大得多,且丢包多由遮挡、传输错误导致。传统TCP无法鉴别拥塞丢包和非拥塞丢包,不分区别地频繁启动拥塞控制算法,降低数据发送速率,将导致TCP无法充分利用可用带宽,减小无人机视频传输的速率,增大传输时延。有必要改进标准TCP协议以提高基于TCP协议的无人机视频传输的性能。网络编码的出现对于解决这一问题给出了一个思路。通过在发送端冗余编码,可以一定程度掩盖链路中出现的丢包,从而提高数据传输的速度,降低时延。
	\par
本文分析了基于TCP协议的无人机视频传输所面临的问题,并指出问题核心是TCP协议在面对有损信道时的低效。对近些年来改善有损信道中TCP性能的相关工作进行了梳理。介绍了网络编码原理,尤其是随机线性网络编码。对于batch-coding和pipeline-coding进行了比较,结合无人机视频传输应用场景,分析了其优缺点。在以上基础上,设计实现了流水线编码TCP协议,并将其移植入嵌入式设备。在模拟丢包网络中,对流水线编码TCP进行性能测试,与TCP-Reno协议进行性能对比。在了解编码TCP协议不足的基础上,对其做了进一步改进,以更好地适应无人机视频传输应用场景。提出了前向重传机制以应对一个RTO内的连续丢包;提出了一种新的自适应冗余度算法以应对网络的变化;设计补偿重传机制以降低解码端解码矩阵维度和减小解码时延;改进编码系数的设定以降低首部开销。最后,本文搭建了无人机视频传输实验平台,验证改进后的编码TCP协议对无人机视频传输性能的提高。
	\par
	
}
{无人机视频传输;传输层协议;编码;重传;自适应冗余;丢包} 	% 中文关键词输入区,关键词之间用“;”分开,不超过五个