%---------------------------------------------------------------------------------
%                西南交通大学研究生学位论文:英文摘要
%---------------------------------------------------------------------------------

% 英文摘要输入区
\eabstract
{
	Unmanned Aerial Vehicles(UAVs) generally work in outdoor airspace, 
	with the characteristics of high mobility,
	which determines that it's often faced with large-scale fading channel environment. 
	Packets corrupt and drop frequently.
	Therefore, UAV video transmission based on the transport layer protocol(TCP) is faced with the problem of limited throughput and large latency. 
	These problems are largely due to the design of the TCP itself.
	\par
	In TCP, 
	the receiver tells the sender which packets are lost through the feedback ACKs. 
	The sender then retransmits the corresponding packets accordingly.
	For UAV video transmission and other real-time applications,
	the feedback delay is too large.
	Not only that, TCP treats the packet loss as the signs of link congestion, and then starts the corresponding congestion control algorithm, reduces the transmission rate.
	Taking into account the very low bit error rate in the wired network and congestion is the main reason for packet loss,
	this mechanism is applicable.
	However, for UAVs, due to weather,obstacles,multipath interference and other factors, TCP packet loss rate is much larger than in wired networks, and packet loss is mainly caused by the block and packet corruption.
	Traditional TCP can not identify congestion packet loss and non-congestion packet loss, and starts the congestion algorithm indiscriminately, which will reduce the transmission rate, cause TCP unable to make full use of available bandwidth, reduce the rate of UAV video transmission and increase the transmission delay.
	It is necessary to improve the standard TCP protocol to improve the performance of TCP-based UAV video transmission.
	The emergence of network coding gives us a way to solve this problem.
	Through redundant coding at the sender side, to a certain extent,the packet loss can be masked, thereby increasing the speed of data transmission and reducing the delay.
	\par
	This paper analyzes the problems of TCP-based UAV video transmission, and verifies that the core of the problems is the inefficiency of the TCP in the face of a lossy channel. 
	Recent work to improve the performance of TCP in lossy channel is sorted out.
	The principle of network coding is introduced,especially the linear network coding.
	Batch-coding and pipeline-coding are compared.
	For UAV video transmission application scenarios, the advantages and disadvantages are analyzed.
	On the basis of above, the pipelined coding TCP protocol is designed,implemented and transplanted into embedded devices.
	In simulated lossy network,the performance of coding TCP and standard TCP-Reno are tested and compared.
	On the basis of understanding the disadvantages of coding TCP,this paper makes further improvements to make coding TCP better adapt to UAV video transmission applications.
	A forward retransmission mechanism is proposed to deal with burst packet loss within an RTO;
	A new adaptive redundancy algorithm is proposed to deal with the fluctuation of the network;
	Improve the transmission timing of the redundant packet to reduce the average decoding delay;
	Design a compensation retransmission mechanism to reduce the dimension of the decoding matrix and the decoding delay;
	Redesign the rule of setting the coefficients to reduce the overhead.
	Finally,this paper builds a UAV video transmission experimental platform to verify the effectiveness of enhanced coding TCP for improving the UAV video transmission.
}
{UAV Video Transmission;Transport Protocol;Network Coding; Retransmission;Adaptive Redundancy;Packet Loss}	% 英文关键词输入区,关键词之间用“; ”分开,不超过五个